Windows, one of the most popular OS, has a component called Telemetry. It collects information from the system with the goal of analyzing and fixing software \& hardware problems, improving the user experience, among others. The kind of information that can be obtained by this component is partially configurable in four different levels: security, basic, enhanced and full, being ”security” the level where less information is gathered and ”full” the opposite case.  

How Telemetry stores/process/administrates the information extracted? It employs a widely used framework called Event Tracer for Windows (ETW) [4]. Embedded not only in userland application but also in the kernel modules, the ETW framework has the goal of providing a common interface to log events and
therefore help to debug and log system operations, by instrumenting it.

In this work, we are going to analyze a part of the Windows kernel to better understand how Telemetry works from an internal perspective. This work will make windows analysts, IT admins or even windows users, more aware about the functionality of the Telemetry component helping to deal with privacy issues, bug fixing, knowledge of collected data, etc. Our analysis implies performing reverse engineering [5],[6] on the Telemetry component, which involves chal- lenging processes such as kernel debugging, dealing with undocumented kernel internal structures, reversing of bigger frameworks (i.e: ETW), binary libraries which lack symbols, etc. As part of the analysis we will also develop an in depth comparison between the differences among each level of Telemetry, stressing the contrast in the amount of events written, verbosity of information, etc. Finally, we will study how the communication between the Windows instance and the Microsoft backend servers is carried out.
