\documentclass[11pt,a4paper,twoside]{tesis}
% SI NO PENSAS IMPRIMIRLO EN FORMATO LIBRO PODES USAR
%\documentclass[11pt,a4paper]{tesis}

\usepackage{graphicx}
\usepackage[utf8]{inputenc}
\usepackage[spanish]{babel}
\usepackage[left=3cm,right=3cm,bottom=3.5cm,top=3.5cm]{geometry}

\begin{document}

%%%% CARATULA

\def\autor{Pablo Agustín Artuso \\ LU: 282/11 \\ artusopablo@gmail.com}
\def\tituloTesis{Analyzing Windows Telemetry component}
\def\runtitulo{Analyzing Windows Telemetry component}
\def\runtitle{}
\def\director{Rodolfo Baader $<$rbaader@dc.uba.ar$>$}
\def\codirector{Aleksandar Milenkoski $<$amilenkoski@ernw.de$>$}
\def\lugar{Buenos Aires, 2018}
\input{caratula}

\section*{Abstract (english version)}
\section*{Abstract (spanish version)}
\section*{Greetings}
\section*{Motivation}

\section*{Table of content}

\section*{Introduction}
Explain that this study was carried out in a particular version of windows (1607) 64 bits Enterprise, which lacked of different things (for instance: documentation for different events)
\subsection*{Basic concepts}
\subsubsection*{Reverse Engineering}
Key concepts about what RE means, from a general perspective, how it should tackled which tools are usually there.  64 bits, calling convention .. 
\subsubsection*{Debugging}
Explaniation of the concept of debugging, what is the different with doing static RE. Some words specifically for KERNEL debugging. 
\subsection*{Tools}
\subsubsection*{IDA pro}
Introduction to IDA pro. Explaniation of what it is and how it works.
\subsubsection*{WinDBG}
Introduction to WINDBG pro. Explaniation of what it is and how it works.
\subsubsection*{YARA ? }
\subsubsection*{XPERF? }

\subsection*{Windows components}
\subsubsection*{Event Tracing for Windows}
Complete explainiation of how it works due to its importance for the rest of the analysis. Different components: session, providers, consumers ,etc .
\subsubsection*{Telemetry}
Full description of the different features / characteristic which are involved in this analysis. Explaniation of how the worflow of the data is followed.


\section*{Previous Work}
Some lines about previous works in this topic. Most of them focused on just analysis from traffic / documentation. 

\section*{Analysis}
Specifying the goals we want to achieve and how to get them. 
\subsection*{Understanding how Telemtry makes use of ETW}
How ETW works internally: Functions that are being called, 
\subsubsection*{When and how providers are registered}
\subsubsection*{How writes are carried out}
\subsubsection*{Relation between ETW session and ETW providers}
\subsubsection*{Identifying the buffers}
\subsubsection*{Provider GUID vs Group Provider GUID}
\subsubsection*{Checking correctness of logged events}
\subsubsection*{Automatization of event logging}
\subsection*{Service isolation}
\subsection*{Triggers}
\subsubsection*{searching for new triggers} YARA

\subsection*{Difference among configuration levels of telemtry}
\subsection*{Analysis of sent data over the channel to Microsfot backend services}



\section*{Results}
\section*{Conclusions}

%
%
%
%%%%% ABSTRACTS, AGRADECIMIENTOS Y DEDICATORIA
%\frontmatter
%\pagestyle{empty}
%\section*{Abstract (spanish version)}
%\input{abs_esp.tex}
%
%\cleardoublepage
%\section*{Abstract (english version)}
%Windows, one of the most popular OS, has a component called Telemetry. It collects information from the system with the goal of analyzing and fixing software \& hardware problems, improving the user experience, among others. The kind of information that can be obtained by this component is partially configurable in four different levels: security, basic, enhanced and full, being ”security” the level where less information is gathered and ”full” the opposite case.  

How Telemetry stores/process/administrates the information extracted? It employs a widely used framework called Event Tracer for Windows (ETW) [4]. Embedded not only in userland application but also in the kernel modules, the ETW framework has the goal of providing a common interface to log events and
therefore help to debug and log system operations, by instrumenting it.

In this work, we are going to analyze a part of the Windows kernel to better understand how Telemetry works from an internal perspective. This work will make windows analysts, IT admins or even windows users, more aware about the functionality of the Telemetry component helping to deal with privacy issues, bug fixing, knowledge of collected data, etc. Our analysis implies performing reverse engineering [5],[6] on the Telemetry component, which involves chal- lenging processes such as kernel debugging, dealing with undocumented kernel internal structures, reversing of bigger frameworks (i.e: ETW), binary libraries which lack symbols, etc. As part of the analysis we will also develop an in depth comparison between the differences among each level of Telemetry, stressing the contrast in the amount of events written, verbosity of information, etc. Finally, we will study how the communication between the Windows instance and the Microsoft backend servers is carried out.
 % OPCIONAL: comentar si no se quiere
%
%\cleardoublepage
%\center \  \section*{Greetings}
%\input{agradecimientos.tex} % OPCIONAL: comentar si no se quiere
%
%\cleardoublepage
%\tableofcontents
%
%\mainmatter
%\pagestyle{headings}
%
%%%%% ACA VA EL CONTENIDO DE LA TESIS
%
%\chapter{La guerra de las galaxias}
%\section{Infancia y juventud}
%{\begin{small}%
%\begin{flushright}%
%\it
%There's nothing for me now.
%I want to learn the ways of\\ the Force and become a Jedi like my father. \\
%--Luke Skywalker
%\end{flushright}%
%\end{small}%
%\vspace{.5cm}}
%
%\section{Rescate de la princesa}
%{\begin{small}%
%\begin{flushright}%
%\it
%Here's where the fun begins!\\
%--Han Solo
%\end{flushright}%
%\end{small}%
%\vspace{.5cm}}
%
%\section{Sacrificio y victoria}
%{\begin{small}%
%\begin{flushright}%
%\it
%This will be a day long remembered.\\ It has seen the end of Kenobi.\\ It will soon see the end of the Rebellion.\\
%--Darth Vader
%\end{flushright}%
%\end{small}%
%\vspace{.5cm}}
%
%%% ...
%\chapter{El imperio contraataca}
%\chapter{El regreso del Jedi}
%
%%%%% BIBLIOGRAFIA
%\backmatter
%%\bibliography{tesis}

\end{document}
